\begin{lstlisting}[language=C++, caption={Polimorfismo 1}]
/**
* Polimorfismo
* 
* cuando.-
*  Dos instancias de dos subclases de una misma clase base,
* al recibir un mismo mensaje, reaccionan de diferente manera
* 
* mensaje == metodo -> cpp
* 
* 
* -> Se implementa con sobreescritura
*/


#include <string>

using namespace std;

class person
{
    string name;
    size_t id;

    public:
    person(const string& name, size_t id)
    :name{name}, id{id}
    {

    }

    // virtual para polimorfismo
    // virtual void print() const
    virtual void print() const
    {
        printf("(%lu) %s", id, name.c_str());
    }
    
    virtual ~person()
    {
        puts("bye");
    }
};

class student: public person
{
    size_t score;
    public:
    student(const string& name, size_t id, size_t score)
    :person{name, id},
    score{score}
    {
    }

    ~student()
    {
        puts("Bye student");
    }

    void print() const override
    {
        // person::print(); // HIDING esconde el metodo de la clase padre
        printf(" Score %lu ", score);
    }
};

int main()
{
    person* c = new person{"Juan", 123};
    c->print();
    delete c;

    student* d = new student{"Omar", 456, 70};
    d->print();
    delete d;

    puts("******");
    person* e = new student("Axel", 32434, 60);
    e->print();
    delete e; // llama al destructor de la clase base
}

/**
* Si se pone virtual entonces se crea un puntero extra en tu objecto
* que se llama VTABLE
* que son tablas de elementos virtuales y crece en memoria
* Y llamar a mtodos virtuales cuesta
* De un objeto->vtable->ejecutaMetodo
* el compilador no tiene poder sobre el virtual
* 
* 
*/
\end{lstlisting}

\begin{lstlisting}[language=C++, caption={Polimorfismo 2}]
#include <cstdio>

struct Animal
{
    virtual ~Animal(){};
    virtual void comunicarse() const = 0;
    // si se iguala a 0 entonces es un metodo abstracto;
    // pero se llaman metodos virtuales puros
    // Por lo que hace que toda la clase es abstracta
};

struct Felino: Animal
{
    protected:
    Felino(){};
};

struct Perro: Animal
{
    void comunicarse() const override
    {
        puts("Woof Woof");
    }
};

struct Vaca: Animal
{
    void comunicarse() const override
    {
        puts("Muuuuuhhh");
    }
};

struct Gato:Felino
{
    /* data */
};


void hacerAlgo(const Animal& a)
{
    a.comunicarse();
}

int main()
{
    Perro p;
    Vaca v;
    // Animal a;
    hacerAlgo(p);
    hacerAlgo(v);
}


// Las clases ABC
// Abstract Base Class
// las cuales solo tienen firmas
// menos el destructor
\end{lstlisting}
