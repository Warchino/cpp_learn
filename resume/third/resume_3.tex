% Aqui se define caracteristicas del documento
% como tamanio de las letras, el papel
% Lo que esta entre llaves {article} significa que el documento es de tipo articulo
\documentclass[11pt,letterpaper]{article}


% Son como las dependencias que utiliza latex para hacer el documento
\usepackage[top=1in,bottom=1in,left=1in,right=1in]{geometry}
\usepackage{xcolor}
\usepackage{listings}
\usepackage[most]{tcolorbox}
\usepackage[spaces,hyphens]{url}
\usepackage[colorlinks,allcolors=blue]{hyperref}
\usepackage{pdfpages}
\usepackage{xcolor}
\usepackage{textcomp}
\usepackage{amsmath}
\usepackage{graphicx}


% define color es como un typedef en c
\definecolor{Darkgreen}{rgb}{0,0.4,0}
\definecolor{listinggray}{gray}{0.9}
\definecolor{lbcolor}{rgb}{0.9,0.9,0.9}

% Aqui se modifica el comando lstset con ciertas caracteristicas
% para mostrar codigo
\lstset{
backgroundcolor=\color{lbcolor},
    tabsize=4,    
    language=[GNU]C++,
        basicstyle=\scriptsize,
        upquote=true,
        aboveskip={1.5\baselineskip},
        columns=fixed,
        showstringspaces=false,
        extendedchars=false,
        breaklines=true,
        prebreak = \raisebox{0ex}[0ex][0ex]{\ensuremath{\hookleftarrow}},
        frame=single,
        numbers=left,
        showtabs=false,
        showspaces=false,
        showstringspaces=false,
        identifierstyle=\ttfamily,
        keywordstyle=\color[rgb]{0,0,1},
        commentstyle=\color[rgb]{0.026,0.112,0.095},
        stringstyle=\color[rgb]{0.627,0.126,0.941},
        numberstyle=\color[rgb]{0.205, 0.142, 0.73},
%        \lstdefinestyle{C++}{language=C++,style=numbers}’.
}
\lstset{
    backgroundcolor=\color{lbcolor},
    tabsize=4,
  language=C++,
  captionpos=b,
  tabsize=3,
  frame=lines,
  numbers=left,
  numberstyle=\tiny,
  numbersep=5pt,
  breaklines=true,
  showstringspaces=false,
  basicstyle=\footnotesize,
%  identifierstyle=\color{magenta},
  keywordstyle=\color[rgb]{0,0,1},
  commentstyle=\color{Darkgreen},
  stringstyle=\color{red},
  upquote=true
  }

% Aqui empieza el documento tiene su begin y end
\begin{document}

% Aqui empieza el titulo de pagina con ciertas caracteristicas
\begin{titlepage}
    \vspace*{\stretch{1.0}}
    \begin{center}
       \Large\textbf{Resumen Third Cpp}
    \end{center}
    \vspace*{\stretch{2.0}}
 \end{titlepage}

% aca se puede continuar el documento pero para hacerlo mas legible
% se crean nuevos archivos .tex pero los cuales ya no tienen el encabezado 
% que se vio previamente, sino directo va el contenido
\section{Class 1}
\subsection{Templates}
Es una herramienta que nos permite reutilizar un cierto codigo con distintas variables, por ejemplo:
\begin{lstlisting}[language=C++, caption={Templates 1}]
int sum(int a, int b)
{
    return a + b;
}

double sum(double a, double b)
{
    return a + b;
}
\end{lstlisting}
Por esta razon al tener la misma estructura, los templates son versatiles.
\begin{lstlisting}[language=C++, caption={Templates 2}]
// typename == class es lo mismo
template <class T>
T sum(T a, T b)
{
    return a + b;
}
\end{lstlisting}

\subsection{Typename explicito en Template}
Hay casos en que se puede dar que el \textbf{typename o class} sea explicito 
al usar un template, para eso se lo realiza de la siguiente manera.
\begin{lstlisting}[language=C++, caption={'T explicito'}]
auto s = sum<std::string>("hello", "world");
\end{lstlisting}


\subsection{Sobrecarga de operador}
Es posible hacer la sobre carga de un operador de una estructura (struct, class) fuera de esta,
porque es posible que no se pueda modificar la estructura por dentro.

\begin{lstlisting}[language=C++, caption={'Overload'}]
struct U
{
    int n;
    
};
// Sobre carga de operador
U operator+(const U& a, const U& b)
{
    return U{a.n + b.n};
}
\end{lstlisting}

\subsection{Template como herramienta al compilar}
Es posible usar el compilador para calculos muy complejos antes del tiempo de ejecucion,
Por ejemplo:
\begin{lstlisting}[language=C++, caption={'Calculos complejos'}]
template <int N>
void star()
{
    for (size_t i = 0; i < N; i++)
    {
        printf("*");
    }
    puts("");
}
int main()
{
    // Esto lo ejecuta el compilador.
    star<6>();
}
\end{lstlisting}


\subsection{Diferencias entre const y constexpr}
La diferencia abismal entre ambos esta en que tiempo se ejecuta:
\begin{itemize}
    \item \textbf{const} se ejecuta en tiempo de ejecucion.
    \item \textbf{constexpr} lo hace en tiempo de compilacion.
\end{itemize}

\subsection{Ventajas y desventajas}
Se hizo un ejemplo de LinkedList[1].
\subsubsection{Ventajas}
\begin{itemize}
    \item Codigo optimo para el tipo de dato usado.
    \item Solo genera codigo de las funciones que usa.
\end{itemize}
\subsubsection{Desventajas}
\begin{itemize}
    \item El compilador tiene que conocer la implementacion, es por eso que si se usa templates,
    todo el codigo debe estar en el header (.h).
\end{itemize}
\section{Class 2}

\subsection{Especializacion de Templates}
Puede suceder en ciertos casos, los typos T del template,
se comportan de manera distinta, como la impresion de un double a un integer 
es diferente, por ello se crearon las Especializaciones.
\begin{lstlisting}[language=C++, caption={'Especializacion 1'}]
    template <typename T>
    void f(T n)
    {
        printf("%d\n", n);
    }


    // Especializacion
    template <>
    void f<double> (double n)
    {
        printf("%f\n", n);
    }
\end{lstlisting}
 
\subsection{Especializacion de clase}
Hasta es posible realizar la especializacion de una clase.
\begin{lstlisting}[language=C++, caption={'Especializacion de clase'}]
    template <class T>
    class Wrapper
    {
        T value;
    public:
        Wrapper(const T& x) : value{x}
        {}
        const T& get() const
        {
            return value;
        }
    };

    struct Point
    {
        int x, y;
    };

    // Especializacion de clase
    template <>
    class Wrapper<Point>
    {
        Point p;
    public:
        Wrapper(const Point& p) : p{p} {}
        void print() const
        {
            printf("(%d %d)\n", p.x, p.y);
        }
    };
\end{lstlisting}
Donde se puede denotar que la especializacion se encuentra al lado del
nombre de la clase, y el keyword \textbf{template} sin argumentos.
\begin{lstlisting}[language=C++, caption={'Especializacion'}]
    template <>
    class Wrapper<Point>
    {...}
\end{lstlisting}

\subsection{Especializacion parcial}
Es interesante porque tambien la especializacion se puede dar de manera parcial.
\begin{lstlisting}[language=C++, caption={'Especializacion parcial'}]
    template <class T, class U>
    struct Q
    {
        T a;
        U b;
        auto sum() const -> decltype(a + b)
        {
            return a + b;
        }
    };

    // Aca solo se especializa el parametro U por un string.
    template <typename T>
    struct Q<T, std::string>
    {
        T a;
        std::string b;

        void print() const
        {
            puts((b + std::to_string(a)).data());
        }
    };
\end{lstlisting}

\subsection{Decltype}
El keyword \textbf{decltype} devuelve el tipo de dato de un elemento.
\begin{lstlisting}[language=C++, caption={decltype}]
    int a = 2;
    decltype(a) b = a;
\end{lstlisting}



\section{Class 3}

\subsection{Function Object}
Es objeto que se comporta como una funcion. \\
Sirve cuando se tiene un objecto en el que se necesita procesar ciertos
metodos y atributos los cuales se enceuntran almacenados y cambian en el tiempo. \par
Practicamente es una sobrecarga al operador \textit{\textbf{()}}.
\begin{lstlisting}[language=C++, caption={Functor}]
    struct P
    {
        void operator()(int a, int b) const
        {
            printf("%d\n", a+b);
        }
    };
\end{lstlisting}


\subsection{Funcion Lambda o Anonima}
Una funcion lambda es una funcion declara en-linea, que puede ser \textbf{functor} 
o simplemente un puntero a funcion.
\begin{itemize}
    \item Si captura variables dentro del \textbf{[]}, es un functor.
    \item Si no, entonces es simplemente un puntero a funcion.
\end{itemize}
\begin{lstlisting}[language=C++, caption={Funcion lambda}]
    f([](int x, int y){return x+y;}, 50, 50);
\end{lstlisting}

\subsection{Policies}
Policies es una forma de evitar el polimorfismo
se evita la herencia, pero se realiza en tiempo de compilacion por lo que se vuelve mas
lento al compilar. \\
Si es necesario agregar nuevos objetos el problema esta en aumentar
los tipos.

\subsubsection{Antes de Policies}
\begin{lstlisting}[language=C++, caption={Before Policies}]
    struct Animal
    {
        virtual ~Animal(){}
        virtual void hablar() const = 0;
    };

    struct vaca : Animal
    {
        void hablar() const override
        {
            puts("muuuuuu");
        }
    };
\end{lstlisting}

\subsubsection{Despues de Policies}
\begin{lstlisting}[language=C++, caption={Using Policies}]
    template <typename HablarPolicy>
    struct Animal
    {
        HablarPolicy hp;
        void hablar() const
        {
            hp.hablar();
        }
    };

    struct hablar_perro
    {
        void hablar() const
        {
            puts("woof");
        }
    };
    using perro = Animal<hablar_perro>;
\end{lstlisting}


Cabe denotar que el rendimiento del programa mejora con el uso de templates
a comparacion de programacion orientada a objetos.

\subsection{Ejemplo Policies}

\subsubsection*{Caso 1}
Se tiene este tipo de implementacion.
\begin{lstlisting}[language=C++, caption={Solucion no generica}]
    template <typename T>
    struct W
    {
        T* val;
        ~W()
        {
            delete val;
    // Esta implementacion no sirve porque no siempre le van a llegar 
    // punteros por new, porque es posible que le lleguen punteros por malloc
        }
    };
\end{lstlisting}

\subsubsection*{Solucion 1}

\begin{lstlisting}[language=C++, caption={Policies example}]
    // Se creo un template para los diferentes tipo de liberar memoria
    template <typename T>
    struct deleter
    {
        void release(T* x)
        {
            delete x;
        }
    };
    // especializacion
    template <>
    struct deleter<char>
    {
        void release(char* x)
        {
            free(x);
        }
    };


    template <typename T, typename Deleter = deleter<T> >
    struct W
    {
        T* val;
        ~W()
        {
            Deleter d;
            d.release(val);
        }
    };
\end{lstlisting}

\section{Class 4}
\end{document}