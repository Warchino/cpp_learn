% Para crear capitulos a tu documento se utiliza el keyword section{"Nombre del capitulo"}
\section{Class 1}
% Sigue el mismo formato para subcapitulos
\subsection{Templates}
Es una herramienta que nos permite reutilizar un cierto codigo con distintas variables, por ejemplo:

% lstlisting es un comando para poner codigo pero
% Este es el comando personalizado que se vio en el header del documento
% donde se puede poner un lenguaje en especifico y este lo colorea de una 
% manera mas pintuda, tiene su begin y su end 
\begin{lstlisting}[language=C++, caption={Templates 1}]
int sum(int a, int b)
{
    return a + b;
}

double sum(double a, double b)
{
    return a + b;
}
\end{lstlisting}
Por esta razon al tener la misma estructura, los templates son versatiles.
\begin{lstlisting}[language=C++, caption={Templates 2}]
// typename == class es lo mismo
template <class T>
T sum(T a, T b)
{
    return a + b;
}
\end{lstlisting}

\subsection{Typename explicito en Template}
Hay casos en que se puede dar que el \textbf{typename o class} sea explicito 
al usar un template, para eso se lo realiza de la siguiente manera.
\begin{lstlisting}[language=C++, caption={'T explicito'}]
auto s = sum<std::string>("hello", "world");
\end{lstlisting}


\subsection{Sobrecarga de operador}
Es posible hacer la sobre carga de un operador de una estructura (struct, class) fuera de esta,
porque es posible que no se pueda modificar la estructura por dentro.

\begin{lstlisting}[language=C++, caption={'Overload'}]
struct U
{
    int n;
    
};
// Sobre carga de operador
U operator+(const U& a, const U& b)
{
    return U{a.n + b.n};
}
\end{lstlisting}

\subsection{Template como herramienta al compilar}
Es posible usar el compilador para calculos muy complejos antes del tiempo de ejecucion,
Por ejemplo:
\begin{lstlisting}[language=C++, caption={'Calculos complejos'}]
template <int N>
void star()
{
    for (size_t i = 0; i < N; i++)
    {
        printf("*");
    }
    puts("");
}
int main()
{
    // Esto lo ejecuta el compilador.
    star<6>();
}
\end{lstlisting}


\subsection{Diferencias entre const y constexpr}
La diferencia abismal entre ambos esta en que tiempo se ejecuta:

% itemize
% es un comando que sirve para poner listas con punto
% para crear un nuevo item se utiliza el comando \item
\begin{itemize}
    \item \textbf{const} se ejecuta en tiempo de ejecucion.
    \item \textbf{constexpr} lo hace en tiempo de compilacion.
\end{itemize}

\subsection{Ventajas y desventajas}
Se hizo un ejemplo de LinkedList[1].
\subsubsection{Ventajas}
\begin{itemize}
    \item Codigo optimo para el tipo de dato usado.
    \item Solo genera codigo de las funciones que usa.
\end{itemize}
\subsubsection{Desventajas}
\begin{itemize}
    \item El compilador tiene que conocer la implementacion, es por eso que si se usa templates,
    todo el codigo debe estar en el header (.h).
\end{itemize}