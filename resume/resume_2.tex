\documentclass[11pt,letterpaper]{article}
\usepackage[top=1in,bottom=1in,left=1in,right=1in]{geometry}
\usepackage{xcolor}
\usepackage{listings}
\usepackage[most]{tcolorbox}
\usepackage[spaces,hyphens]{url}
\usepackage[colorlinks,allcolors=blue]{hyperref}
\usepackage{pdfpages}
\usepackage{xcolor}
\usepackage{textcomp}

\definecolor{Darkgreen}{rgb}{0,0.4,0}
\definecolor{listinggray}{gray}{0.9}
\definecolor{lbcolor}{rgb}{0.9,0.9,0.9}
\lstset{
backgroundcolor=\color{lbcolor},
    tabsize=4,    
%   rulecolor=,
    language=[GNU]C++,
        basicstyle=\scriptsize,
        upquote=true,
        aboveskip={1.5\baselineskip},
        columns=fixed,
        showstringspaces=false,
        extendedchars=false,
        breaklines=true,
        prebreak = \raisebox{0ex}[0ex][0ex]{\ensuremath{\hookleftarrow}},
        frame=single,
        numbers=left,
        showtabs=false,
        showspaces=false,
        showstringspaces=false,
        identifierstyle=\ttfamily,
        keywordstyle=\color[rgb]{0,0,1},
        commentstyle=\color[rgb]{0.026,0.112,0.095},
        stringstyle=\color[rgb]{0.627,0.126,0.941},
        numberstyle=\color[rgb]{0.205, 0.142, 0.73},
%        \lstdefinestyle{C++}{language=C++,style=numbers}’.
}
\lstset{
    backgroundcolor=\color{lbcolor},
    tabsize=4,
  language=C++,
  captionpos=b,
  tabsize=3,
  frame=lines,
  numbers=left,
  numberstyle=\tiny,
  numbersep=5pt,
  breaklines=true,
  showstringspaces=false,
  basicstyle=\footnotesize,
%  identifierstyle=\color{magenta},
  keywordstyle=\color[rgb]{0,0,1},
  commentstyle=\color{Darkgreen},
  stringstyle=\color{red},
  upquote=true
  }
\newtcblisting{commandshell}{colback=black,colupper=white,colframe=yellow!75!black,
listing only,listing options={language=sh},
every listing line={\textcolor{green}{\small\ttfamily\bfseries RodrigoMenacho \$> }}}


% Packages
\usepackage{amsmath}
\usepackage{graphicx}

% Document
\begin{document}
\begin{center}
    \textbf{Resumen C++}
\end{center}

\section{Class 1}
\subsection{Mejoras sobre C}
\begin{itemize}
    \item Existe la sobrecarga de funciones sin perder ningun recurso en memoria
    \item El include de las librerias no termina en .h \textbf{cstdio = stdio.h}
\end{itemize}
\subsection{Namespace}
Es un contenedor de identificador.
\begin{itemize}
    \item Variable
    \item Funciones
    \item Structs
    \item Clases
    \item Tipos Enum
    \item Namespaces
\end{itemize}
Desde Cpp-17 existes Namespaces anidados: \\
A::B::C
\subsection{Referencias}
\begin{quote}
    Son punteros disfrazados de ovejas.
\end{quote}
Porque internamente son punteros.
Limitaciones: 
\begin{itemize}
    \item Inmutables
    \begin{itemize}
        \item Establecida la referencia y su referido esa relacion no se puede cambiar
    \end{itemize}
    \item No hay aritmetica a referencias 
    \item Cuando se accede a la referencia se accede al valor del referido
    \item No hay referencias a NULL
\end{itemize}
\subsection{Scoped Enums}
\begin{lstlisting}[language=C++, caption={Scoped enums}]
enum class Priority
{
	RED, BLUE, BLACK;
};

enum class Color
{
	RED, GREEN, BLUE;
};
\end{lstlisting}

\section{Class 2}
\subsection{Clase y Struct}
En la clase todo es privado por defecto.\\
En el struct todo es publico por defecto.
\subsection{Const al final del metodo}
El const al final dice que no se modificara los atributos de la clase
y solo llama a metodos constantes.
\begin{lstlisting}[language=C++, caption={method() const}]
void mostrar() const
{
    printf("(%zu) %s %s\n",ci, ln, fn);
}
\end{lstlisting}
\subsection{This}
Es un puntero que representa la instancia.

\begin{lstlisting}[language=C++, caption={this->}]
Person(const char* fn, const char* ln, size_t ci)
{
    strcpy(this->fn, fn);
    strcpy(this->ln, ln);
    this->ci = ci;
}
\end{lstlisting}

\subsection{Inicializacion Uniforme}
Es por eso que se crea un objeto con llaves {} para que no se raye el compilador cuando lee parentesis, 
porque puede compilar como si fuese un metodo.

\begin{lstlisting}[language=C++, caption={Inicializacion con llaves}]
Person q{"Julian", "Assange", 1961};
\end{lstlisting}

\section{Class 3}
\subsection{Initialization List}
Cuando podemos inicializar los atributos de un constructor usando llaves dentro de la clase.
\begin{lstlisting}[language=C++, caption={Inicializacion con llaves en constructor}]
student(size_t id, const char *name)
    : id{id} //Initialization list
// forma preferida para inicializar atributos
{
    auto len = strlen(name);
    this->name = (char *)malloc(len + 1);
    memcpy(this->name, name, len + 1);
}
\end{lstlisting}

\subsection{Regla de 3}
Si existe un malloc es necesario que se reimplemente el destructor, constructor y el operador igual “=”
de una clase o struct para evitar problemas en compilacion.

\subsection{RAII}
\begin{quotation}
    \textbf{Resource adquisition is initialization}
\end{quotation}
Es decir, es una caracteristica en C++ donde el elemento que fue creado dentro de un cierto Scope
se destruye al salir de este scope.

\subsection{Copia}
\begin{lstlisting}[language=C++, caption={Copia}]
auto student z{2, "Antonio jose de sucre"};
auto q = z; // copy
\end{lstlisting}
\subsection{Asignacion}
\begin{lstlisting}[language=C++, caption={Asignacion}]
student r{666, "norbert"};
r = q; // Asignacion
\end{lstlisting}

\section{Class 4}
Se implemento la clase ztring xD.

\begin{lstlisting}[language=C++, caption={Clase Ztring}]
#include <cstdio>
#include <cstring>
#include <cstdlib>

constexpr size_t MAX = 16;

class ztring
{
    char* chars;
    // SSO: small string optimization
    char szo[MAX];
    size_t len;
    public:
    ztring(const char* s = "")
    :len{strlen(s)}
    {
        set_string(s);
    }
    const char* data() const
    {
        return len < MAX ? szo : chars;
    }
    ~ztring()
    {
        if (len >= MAX)
        {
            free(chars);
        }
    }
    ztring(const ztring& s)
    :len{s.len}
    {
        set_string(s.data());
    }

    ztring& operator=(const ztring& src)
    {
        if (this != &src)
        {
            return *this;
        }
        this->~ztring();
        len = src.len;
        set_string(src.data());
        return *this;
    }

    ztring operator+(const ztring& s) const
    {
        auto nlen = len + s.len;
        ztring ns;
        ns.len = nlen;
        if (nlen >= MAX)
        {
            ns.chars = (char*) malloc(nlen+1);
        }
        char* str = nlen < MAX ? ns.szo : ns.chars;
        memcpy(str, data(), len);
        memcpy(str + len, s.data(), s.len+1);
        return ns;

    }

    ztring& operator+=(const ztring& src)
    {
        auto nlen = len + src.len;
        if (len >= MAX)
        {
            chars = (char*) realloc(chars, nlen+1);
            memcpy(chars + len, src.data(), src.len+1);
            len = nlen;
            return *this;
        }

        if (nlen >= MAX)
        {
            chars = (char*) malloc(nlen+1);
            memcpy(chars, data(), len);
            memcpy(chars + len, src.data(), src.len+1);
            len = nlen;
            return *this;
        }
        memcpy(szo + len, src.data(), src.len+1);
        len = nlen;
        return *this;
    }

    private:
    void set_string(const char* s)
    {
        if (len >= MAX)
        {
            chars = (char*) malloc(len+1);
            memcpy(chars, s, len+1);
        }else
        {
            memcpy(szo, s, len+1);
        }
    }

};

\end{lstlisting}


\section{Class 5}
\subsection{Herencia}
La herencia se lo realiza con los \textbf{:} y se especifica
si es public, private or protected.
\begin{lstlisting}[language=C++, caption={Herencia}]
// si es private todo lo que se hereda, es privado hacia afuera
class student: public person 
{
}
\end{lstlisting}

\subsection{Hiding}
Cuando se da la herencia directamente no es posible el polimorfismo por lo cual
se da el siguiente caso donde se esconde el metodo del padre.
\begin{lstlisting}[language=C++, caption={Hiding}]
void print() const
{
    person::print(); // HIDING esconde el metodo de la clase padre
    printf(" Score %lu ", score);
}
\end{lstlisting}

\subsection{New}
El keyword new es una palabra reservada la cual nos permite crear objetos
del tipo punteros, el cual realiza el pedido de memoria en el HEAP de toda una 
clase o struct. \par
A su vez llama al constructor de la clase. 
\begin{lstlisting}[language=C++, caption={New}]
// auto* q = (point*) malloc(sizeof(point)); 
// no llama al constructor por lo que no sirve

point* q = new point(6, 9); // Pide en el HEAP pero es mas lento
// ***desventajas***
// No es posible usar aritmetica de punteros
// Ni realloc
\end{lstlisting}
Por cada \textbf{new} debe existir un \textbf{delete}, analogamente en c
por cada \textbf{malloc} existe un \textbf{free} 

\subsection{new[]}
Es posible inicializar array de objetos, indicando el numero de objetos que 
tendra el array.
\begin{lstlisting}[language=C++, caption={New[ ]}]
point* ps = new point[3];
// operador new[]
// std::bad_alloc es porque no encuentra espacio en memoria
ps[0].set_values(9,5);
ps[1].set_values(2,1);
ps[2].set_values(0,4);
for (size_t i = 0; i < 3; i++)
{
    ps[i].print();
}
delete[] ps;
\end{lstlisting}
Por cada \textbf{new[ ]} existe un \textbf{delete[ ]}.
\begin{quote}
    \textbf{NO OLVIDAR EL DELETE[ ]}
\end{quote}

\subsection{POD - Plain Old Data}
Hace referencia a:
\begin{itemize}
    \item Tipo primitivo
    \item Array de pods
    \item Structs de pods
    \item Union de pods
\end{itemize}
En el cual, para todos estos PODs es posible usar malloc, realloc and free.

\newpage
\appendix
\section{Ejemplos de Punteros y Referencias }
By Carlos, Marcela, Dani, Karen.
\begin{lstlisting}[language=C++, caption={Examples ref and pointers}]
#include <cstdio>
#include <cstdlib>

int main()
{
    puts("*******Empezamos con punteros**********");
    int b = 2;
    int* a = &b;
    printf("El valor al que apunta a (*a=b): %d \n",*a);
    printf("La direccion a la que apunta a que es la de b: %p \n",a);
    printf("La direccion de b: %p \n",&b);
    printf("La direccion del puntero a: %p \n",&a);

    // int *c = a;
    // printf("%p \n",c);
    // printf("%d \n",*c);

    puts("*********Punteros dobles XD******");

    int** d = &a;
    printf("La direccion a la que apunta a, apuntada por d (*d=a): %p \n",*d);
    printf("El valor al que apunta a, apuntado por d (*d=a) (**d=*a): %d \n",**d);

    int k=5;
    *d = &k;
    printf("Estamos cambiando la direccion a la que apunta a, %p\n",*d);
    printf("La direccion a la que apunta a ahora es la direccion de k:  %p \n",a);
    printf("%d \n",*a);
    
    puts("*********ahora vienen las referencias*******");

    int t = 8;

    int& ref = t;

    printf("El valor de t: %d\n", t);
    printf("El valor de ref es t (ref): %d\n", ref);
    printf("La direccion de t (&t): %p\n", &t);
    printf("La direccion de ref es la dir de t (&ref): %p\n", &ref);
    puts("Por si no te diste cuenta, son lo mismo XP");

    t = 25;

    printf("El valor de t: %d\n", t);
    printf("El valor de ref es t (ref): %d\n", ref);
    printf("La direccion de t (&t): %p\n", &t);
    printf("La direccion de ref es la dir de t (&ref): %p\n", &ref);

    ref = 1000;

    printf("El valor de t: %d\n", t);
    printf("El valor de ref es t (ref): %d\n", ref);
    printf("La direccion de t (&t): %p\n", &t);
    printf("La direccion de ref es la dir de t (&ref): %p\n", &ref);

    puts("*********ahora todo junto*******");

    int *q = &t;

    int& ref2 = *q;
    int*& ref3 = q;

    printf("%p\n", &ref2);
    printf("%d\n", ref2);
    printf("%p\n", ref3);

}
\end{lstlisting}

\section{Code of Class 2}
\begin{lstlisting}[language=C++, caption={class}]
    # include <cstring>
    # include <cstdio>
    
    class Person
    {
        // en C los struct todo es publico por defect
        // en c++ las clases son publicas por defecto
        // Atributos
        char fn[32];
        char ln[32];
        size_t ci;
        
        public:
        // Metodo
        // El const al final dice que no se modificara los atributos de la clase
        // y solo llama a metodos constantes 
        void mostrar() const
        {
            printf("(%zu) %s %s\n",ci, ln, fn);
        }
    
        Person(const char* fn, const char* ln, size_t ci)
        {
            // this representa la instancia y es un puntero
            strcpy(this->fn, fn);
            strcpy(this->ln, ln);
            this->ci = ci;
        }
        Person():Person("", "", 0) // C++11
        {
            
        }
    
        void set_fn(const char* fn)
        {
            strcpy(this->fn, fn);
        }
    
        void set_ln(const char* ln)
        {
            strcpy(this->ln, ln);
        }
    
        void set_ci(size_t ci)
        {
            this->ci = ci;
        }
    };
    /**
        * El constructor tiene el mismo nombre del struct pero sin tipo de retorno
        */
    
    auto main() -> int
    {
        Person p;
        p.mostrar();
        p.set_fn("John");
        p.set_ln("Smith");
        p.set_ci(468952);
        p.mostrar();
        Person q{"Julian", "Assange", 1961};
        // Inicializacion uniforme y es por eso que se crea un objeto con llaves {}
        // para que no se raye el compilador cuando lee parentesis
        q.mostrar();
        int r{25};
        int m{2};
        printf("%d\n", r);
        printf("%d\n", m);
        // Otra manera de inicializar
        auto w = int{22};
    
    }
\end{lstlisting}

\section{Code of Class 3}
\input{a_class3.tex}

\section{Code of Class 4}
\begin{lstlisting}[language=C++, caption={Ztring}]
#include <cstdio>
#include <cstring>
#include <cstdlib>

constexpr size_t MAX = 16;

class ztring
{
    char* chars;
    // SSO: small string optimization
    char szo[MAX];
    size_t len;
    public:
    ztring(const char* s = "")
    :len{strlen(s)}
    {
        set_string(s);
    }
    const char* data() const
    {
        return len < MAX ? szo : chars;
    }
    ~ztring()
    {
        if (len >= MAX)
        {
            free(chars);
        }
    }
    ztring(const ztring& s)
    :len{s.len}
    {
        set_string(s.data());
    }

    ztring& operator=(const ztring& src)
    {
        if (this != &src)
        {
            return *this;
        }
        this->~ztring();
        len = src.len;
        set_string(src.data());
        return *this;
    }

    ztring operator+(const ztring& s) const
    {
        auto nlen = len + s.len;
        ztring ns;
        ns.len = nlen;
        if (nlen >= MAX)
        {
            ns.chars = (char*) malloc(nlen+1);
        }
        char* str = nlen < MAX ? ns.szo : ns.chars;
        memcpy(str, data(), len);
        memcpy(str + len, s.data(), s.len+1);
        return ns;

    }

    ztring& operator+=(const ztring& src)
    {
        auto nlen = len + src.len;
        if (len >= MAX)
        {
            chars = (char*) realloc(chars, nlen+1);
            memcpy(chars + len, src.data(), src.len+1);
            len = nlen;
            return *this;
        }

        if (nlen >= MAX)
        {
            chars = (char*) malloc(nlen+1);
            memcpy(chars, data(), len);
            memcpy(chars + len, src.data(), src.len+1);
            len = nlen;
            return *this;
        }
        memcpy(szo + len, src.data(), src.len+1);
        len = nlen;
        return *this;
    }

    bool operator==(const ztring& s) const
    {
        if (this == &s)
        {
            return true;
        }
        if (len != s.len)
        {
            return false;
        }
        return memcmp(data(), s.data(), len) == 0;
    }

    bool operator>(const ztring& s) const
    {
        size_t auxLen = len > s.len ? len : s.len;
        return memcmp(data(), s.data(), auxLen) > 0;
    }

    bool operator<(const ztring& s) const
    {
        size_t auxLen = len > s.len ? len : s.len;
        return memcmp(data(), s.data(), auxLen) < 0;
    }

    bool operator!=(const ztring& s) const
    {
        return !(*this == s);
    }

    bool operator>=(const ztring& s) const
    {
        size_t auxLen = len > s.len ? len : s.len;
        return memcmp(data(), s.data(), auxLen) >= 0;
    }

    bool operator<=(const ztring& s) const
    {
        size_t auxLen = len > s.len ? len : s.len;
        return memcmp(data(), s.data(), auxLen) <= 0;
    }

    ztring& trim()
    {
        size_t left = 0;
        while (this->data()[left] == ' ')
        {
            left++;
        }
        
        size_t right = len-1;
        while (this->data()[right] == ' ' || right == 0)
        {
            right--;
        }
        size_t nlen = right - left + 1;
        if (len == nlen)
        {

            return *this;
        }
        

        if (nlen >= MAX)
        {
            char* aux = (char*) malloc(len+1);
            memcpy(aux, chars, len + 1);
            chars = (char*) realloc(chars, nlen+1);
            memcpy(chars, aux + left, nlen);
            free(aux);
            len = nlen;
            chars[nlen] = '\0';
            return *this;
        }
        const char* aux = this->data();
        memcpy(szo, aux + left, nlen);
        if (len >= MAX)
        {
            free(chars);
        }
        len = nlen;
        szo[nlen] = '\0';
        return *this;
    }

    private:
    void set_string(const char* s)
    {
        if (len >= MAX)
        {
            chars = (char*) malloc(len+1);
            memcpy(chars, s, len+1);
        }else
        {
            memcpy(szo, s, len+1);
        }
    }

};

int main()
{
    // ztring s = "Hola";
    // puts(s.data());
    // ztring r = "Antonio Jose de Sucre";
    // puts(r.data());

    // auto p = r;
    // puts(p.data());
    // auto x = s;
    // puts(x.data());
    // p = "Segmentation";
    // puts(p.data());
    // // si explicit ztring(const char* s)
    // // entonces la {linea algo} no funciona porque no encuentra el compilador un constrcutor 
    // // de esa forma

    // ztring h = "hello";
    // ztring w = "world";
    // auto hw = h + " " + w;
    // puts(hw.data());

    // auto hw2 = hw + " of C++ segmentation fault";
    // puts(hw2.data());

    // ztring b = "today";
    // b += "is";
    // b += "Tuesday, April 24";
    
    // puts((b += "2019").data());


    ztring z = "               Hello World magic world of narnia   ";

    puts(z.trim().data());
    printf("Length -> %lu\n", strlen(z.data()));

    ztring a{"ab"};
    ztring b{"aBC"};

    puts("---------comparators---------");
    puts(">=");
    puts(a >= b ? "true" : "false");
    puts(strcmp(a.data(), b.data()) >= 0 ? "true" : "false");
    puts(">");
    puts(a > b ? "true" : "false");
    puts(strcmp(a.data(), b.data()) > 0 ? "true" : "false");
    puts("<=");
    puts(a <= b ? "true" : "false");
    puts(strcmp(a.data(), b.data()) <= 0 ? "true" : "false");
    puts("<");
    puts(a < b ? "true" : "false");
    puts(strcmp(a.data(), b.data()) < 0 ? "true" : "false");
    puts("==");
    puts(a == b ? "true" : "false");
    puts(strcmp(a.data(), b.data()) == 0 ? "true" : "false");
    puts("!=");
    puts(a != b ? "true" : "false");
    puts(strcmp(a.data(), b.data()) != 0 ? "true" : "false");
    // operator==
    // operator!=
    // operator>=
    // operator>
    // operator<=
    // operator<
    // trim();
// es posible usar strcmp
    return 0;
}
\end{lstlisting}

\section{Code of Class 5}
\begin{lstlisting}[language=C++, caption={Herencia}]
#include <string>

using namespace std;

class person
{
    string name;
    size_t id;

    public:
    person(const string& name, size_t id)
    :name{name}, id{id}
    {

    }
    void print() const
    {
        printf("(%lu) %s", id, name.c_str());
    }
    
    ~person()
    {
        puts("bye");
    }
};
// si es private todo lo que se hereda, es privado hacia afuera
class student: public person
{
    size_t score;
    public:
    student(const string& name, size_t id, size_t score)
    :person{name, id},
    score{score}
    {
    }

    ~student()
    {
        puts("Bye student");
    }

    void print() const
    {
        person::print(); // HIDING esconde el metodo de la clase padre
        printf(" Score %lu ", score);
    }
};

int main()
{
    // person p{"Jau pepe", 1235423};
    // p.print();
    // puts("");
    
    // student s{"Pablo Palti", 1235, 95};
    // s.print();
    // puts("");

    person* c = new person{"Juan", 123};
    c->print();
    delete c;

    student* d = new student{"Omar", 456, 70};
    d->print();
    delete d;

    puts("******");
    person* e = new student("Axel", 32434, 60);
    ((student*)e)->print();
    delete e; // llama al destructor de la clase base
}
    
\end{lstlisting}

\begin{lstlisting}[language=C++, caption={Herencia 2}]
#include <cstdio>
#include <cstdlib>

class point
{
    int x;
    int y;
    public:
        point(int x = 0, int y = 0)
        :x{x}, y{y}
        {
        }

        void print() const
        {
            printf("(%d,%d)\n", x, y);
        }

        ~point()
        {
            puts("Bye");
        }
        void set_values(int x, int y)
        {
            this->x = x;
            this->y = y;
        }

};

int main()
{
    point p{6,8};
    p.print();
    // auto* q = (point*) malloc(sizeof(point)); // no llama al constructor por lo que no sirve
    point* q = new point(6, 9); // Pide en el HEAP pero es mas lento
    // new desventajas
    // no se puede hacer como aritmetica de punteros
    // tampoco realloc

    q->print();
    // free(q);
    delete q;

    point* ps = new point[3];
    // operador new[]
    // std::bad_alloc es porque no encuentra espacio en memoria
    ps[0].set_values(9,5);
    ps[1].set_values(2,1);
    ps[2].set_values(0,4);
    for (size_t i = 0; i < 3; i++)
    {
        ps[i].print();
    }
    // delete[]
    delete[] ps;
}


// POD: Plain Old Data
// Tipo primitivo
// array de pods
// structs de pods
// union de pods
// para todo esto se puede usar malloc realloc free
\end{lstlisting}



\end{document}