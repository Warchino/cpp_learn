\begin{lstlisting}[language=C++, caption={class}]
    # include <cstring>
    # include <cstdio>
    
    class Person
    {
        // en C los struct todo es publico por defect
        // en c++ las clases son publicas por defecto
        // Atributos
        char fn[32];
        char ln[32];
        size_t ci;
        
        public:
        // Metodo
        // El const al final dice que no se modificara los atributos de la clase
        // y solo llama a metodos constantes 
        void mostrar() const
        {
            printf("(%zu) %s %s\n",ci, ln, fn);
        }
    
        Person(const char* fn, const char* ln, size_t ci)
        {
            // this representa la instancia y es un puntero
            strcpy(this->fn, fn);
            strcpy(this->ln, ln);
            this->ci = ci;
        }
        Person():Person("", "", 0) // C++11
        {
            
        }
    
        void set_fn(const char* fn)
        {
            strcpy(this->fn, fn);
        }
    
        void set_ln(const char* ln)
        {
            strcpy(this->ln, ln);
        }
    
        void set_ci(size_t ci)
        {
            this->ci = ci;
        }
    };
    /**
        * El constructor tiene el mismo nombre del struct pero sin tipo de retorno
        */
    
    auto main() -> int
    {
        Person p;
        p.mostrar();
        p.set_fn("John");
        p.set_ln("Smith");
        p.set_ci(468952);
        p.mostrar();
        Person q{"Julian", "Assange", 1961};
        // Inicializacion uniforme y es por eso que se crea un objeto con llaves {}
        // para que no se raye el compilador cuando lee parentesis
        q.mostrar();
        int r{25};
        int m{2};
        printf("%d\n", r);
        printf("%d\n", m);
        // Otra manera de inicializar
        auto w = int{22};
    
    }
\end{lstlisting}